% Une ligne commentaire débute par le caractère « % »

\documentclass[a4paper]{article}

% Options possibles : 10pt, 11pt, 12pt (taille de la fonte)
%                     oneside, twoside (recto simple, recto-verso)
%                     draft, final (stade de développement)

\usepackage[utf8]{inputenc}   % LaTeX, comprends les accents !
\usepackage[T1]{fontenc}      % Police contenant les caractères français
\usepackage[francais]{babel}  
\usepackage{fullpage}



\usepackage{graphicx}  % pour inclure des images

%\pagestyle{headings}        % Pour mettre des entêtes avec les titres
                              % des sections en haut de page

 \title{  Titre du sujet\\         % Les paramètres du titre : titre, auteur, date
  Projet de programmation 2}          
\author{ Les membres du groupe\\
  \emph{L3 informatique}\\
  Faculté des Sciences\\
Université de Montpellier.}
\date{2021-2022}             



\begin{document}

\maketitle                    % Faire un titre utilisant les données
                              % passées à \title, \author et \date

%\begin{center}               % pour centrer 
 % \includegraphics[scale=1]{}   % insertion d'une image
%\end{center}

\begin{abstract}     % Résumé du travail

  \emph{Description très succinte du problème et des différentes étapes de réalisation}

\end{abstract}
\newpage
%\dominitoc  % initializer les minitoc
\tableofcontents
\newpage
\section*{Introduction} % Commencer une section, * permet de ne pas numéroté l'introduction
\emph{1-2 pages}
	%Présentation du Sujet (1-2 pages) 
\begin{itemize}%pour faire une liste à puces
\item Présenter le problème étudié et le contexte dans lequel le projet se positionne.
\item Motiver l'intérêt du problème étudié par rapport à votre parcours d'études et au monde de l'informatique.
\item	Présenter les différentes approches possibles pour la résolution du problème, et en particulier celle choisie.
\item Donner le cahier des charges détaillé
\end{itemize}


\section{Développements Logiciel : Conception, Modélisation, Implémentation} 
\emph{8 à 15 pages}

Il se peut que la conception du logiciel (3) et l'analyse algorithmique (4) aient un poids différent dans votre projet. Les points décrits ci-dessous seront ainsi plus ou moins développés en fonction du travail réalisé en accord avec l'encadrant. 
\begin{enumerate}%pour faire une liste numérotée
\item Présenter les développements logiciel réalisés dans le cadre du projet.
\item Présenter les principaux modules du logiciel développé dans le cadre du projet. Utiliser le langage UML pour la modélisation : donner le diagramme de cas d'utilisation et le diagramme des classes.
\item Décrire les fonctionnalités de l'interface graphique implémentée (si votre logiciel dispose d'une interface graphique).
\item Décrire le format des données en entrée ou encore les conventions utilisées pour les entrées de vos programmes. Décrire les procédures de lecture et validation des entrées.
\item Statistiques : nombre de modules/composantes/classes/scripts développés. Nombre de lignes de code.
\end{enumerate}

\section{Algorithmes et Structures de Données}
\emph{2 à 3 pages}
\begin{enumerate}%pour faire une liste numérotée
\item 	Présenter les principales structures de données définies dans le cadre du projet.
\item 	Présenter les principaux algorithmes implémentés. En illustrer le fonctionnement avec des exemples. Attention : il s'agit ici de choisir 1 ou 2 algorithmes intéressants, et non pas de présenter tous les algorithmes implémentés.
\item 	Évaluer la complexité théorique en temps des algorithmes présentés
\end{enumerate}

\section{Analyse des résultats }
\emph{2 à 4 pages}
L'analyse expérimentale sera faites dans cette section.   Elle sera plus ou moins importante dans votre projet comme pour les deux points précédent, cette section devra donc être développée en conséquence.

\begin{enumerate}%pour faire une liste numérotée
\item 	Illustrer les performances ainsi que l'efficacité du logiciel implémenté à l'aide de graphiques.
\item Analyser (et comparer, si plusieurs) les performances des solutions implémentées.
\item	Présenter les bancs d'essais (ou les procédures utilisées pour la génération des données) utilisés pour les tests du logiciel.
\end{enumerate}

\section{Gestion du Projet}
\emph{ (1-2 pages)}
\begin{enumerate}%pour faire une liste numérotée
\item 	Présenter la gestion du projet et les documents de planification rédigés (par exemple, le diagramme de Gantt).
\item	Discuter les changements majeurs effectués en cours de projet. 
\end{enumerate}

\section{Bilan et Conclusions}

	Indiquer les fonctionnalités mises en œuvre par rapport au cahier des charges de départ, les points ouvertes.
	Les perspectives du projet décrivent ce que l’on pourrait ajouter, modifier, réécrire afin d’améliorer le projet. 
	
\begin{thebibliography}{9}
\bibitem{texbook}
Donald E. Knuth (1986) \emph{The \TeX{} Book}, Addison-Wesley Professional.

\bibitem{lamport94}
Leslie Lamport (1994) \emph{\LaTeX: a document preparation system}, Addison
Wesley, Massachusetts, 2nd ed.
\end{thebibliography}



\end{document}

